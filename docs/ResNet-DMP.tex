% Options for packages loaded elsewhere
\PassOptionsToPackage{unicode}{hyperref}
\PassOptionsToPackage{hyphens}{url}
%
\documentclass[
]{book}
\usepackage{lmodern}
\usepackage{amsmath}
\usepackage{ifxetex,ifluatex}
\ifnum 0\ifxetex 1\fi\ifluatex 1\fi=0 % if pdftex
  \usepackage[T1]{fontenc}
  \usepackage[utf8]{inputenc}
  \usepackage{textcomp} % provide euro and other symbols
  \usepackage{amssymb}
\else % if luatex or xetex
  \usepackage{unicode-math}
  \defaultfontfeatures{Scale=MatchLowercase}
  \defaultfontfeatures[\rmfamily]{Ligatures=TeX,Scale=1}
\fi
% Use upquote if available, for straight quotes in verbatim environments
\IfFileExists{upquote.sty}{\usepackage{upquote}}{}
\IfFileExists{microtype.sty}{% use microtype if available
  \usepackage[]{microtype}
  \UseMicrotypeSet[protrusion]{basicmath} % disable protrusion for tt fonts
}{}
\makeatletter
\@ifundefined{KOMAClassName}{% if non-KOMA class
  \IfFileExists{parskip.sty}{%
    \usepackage{parskip}
  }{% else
    \setlength{\parindent}{0pt}
    \setlength{\parskip}{6pt plus 2pt minus 1pt}}
}{% if KOMA class
  \KOMAoptions{parskip=half}}
\makeatother
\usepackage{xcolor}
\IfFileExists{xurl.sty}{\usepackage{xurl}}{} % add URL line breaks if available
\IfFileExists{bookmark.sty}{\usepackage{bookmark}}{\usepackage{hyperref}}
\hypersetup{
  pdftitle={ResNet Data Management Plan},
  pdfauthor={ResNet Synthesis Team},
  hidelinks,
  pdfcreator={LaTeX via pandoc}}
\urlstyle{same} % disable monospaced font for URLs
\usepackage{longtable,booktabs}
\usepackage{calc} % for calculating minipage widths
% Correct order of tables after \paragraph or \subparagraph
\usepackage{etoolbox}
\makeatletter
\patchcmd\longtable{\par}{\if@noskipsec\mbox{}\fi\par}{}{}
\makeatother
% Allow footnotes in longtable head/foot
\IfFileExists{footnotehyper.sty}{\usepackage{footnotehyper}}{\usepackage{footnote}}
\makesavenoteenv{longtable}
\usepackage{graphicx}
\makeatletter
\def\maxwidth{\ifdim\Gin@nat@width>\linewidth\linewidth\else\Gin@nat@width\fi}
\def\maxheight{\ifdim\Gin@nat@height>\textheight\textheight\else\Gin@nat@height\fi}
\makeatother
% Scale images if necessary, so that they will not overflow the page
% margins by default, and it is still possible to overwrite the defaults
% using explicit options in \includegraphics[width, height, ...]{}
\setkeys{Gin}{width=\maxwidth,height=\maxheight,keepaspectratio}
% Set default figure placement to htbp
\makeatletter
\def\fps@figure{htbp}
\makeatother
\setlength{\emergencystretch}{3em} % prevent overfull lines
\providecommand{\tightlist}{%
  \setlength{\itemsep}{0pt}\setlength{\parskip}{0pt}}
\setcounter{secnumdepth}{5}
\usepackage{booktabs}
\ifluatex
  \usepackage{selnolig}  % disable illegal ligatures
\fi
\usepackage[]{natbib}
\bibliographystyle{apalike}

\title{ResNet Data Management Plan}
\author{ResNet Synthesis Team}
\date{2021-04-13}

\begin{document}
\maketitle

{
\setcounter{tocdepth}{1}
\tableofcontents
}
\hypertarget{start}{%
\chapter*{Start}\label{start}}
\addcontentsline{toc}{chapter}{Start}

\includegraphics[width=0.75\textwidth,height=\textheight]{static/ResNet-logo.png}

\begin{quote}
suppress Start heading
\end{quote}

\begin{quote}
versioning info
\end{quote}

\begin{quote}
expand authorship
\end{quote}

\hypertarget{quick-links}{%
\section*{Quick Links}\label{quick-links}}
\addcontentsline{toc}{section}{Quick Links}

\hypertarget{publishing-data}{%
\subsection*{Publishing Data}\label{publishing-data}}
\addcontentsline{toc}{subsection}{Publishing Data}

\hypertarget{getting-help}{%
\subsection*{Getting Help}\label{getting-help}}
\addcontentsline{toc}{subsection}{Getting Help}

\hypertarget{introduction}{%
\chapter{Introduction}\label{introduction}}

\hypertarget{fair-guiding-principles}{%
\section{FAIR Guiding Principles}\label{fair-guiding-principles}}

From the \href{https://www.go-fair.org/fair-principles/}{GO FAIR Initiative}:

\begin{quote}
\textbf{Findable}\\
The first step in (re)using data is to find them. Metadata and data should be easy to find for both humans and computers. Machine-readable metadata are essential for automatic discovery of datasets and services, so this is an essential component of the \href{https://www.go-fair.org/fair-principles/fairification-process/}{FAIRification process}.

\href{https://www.go-fair.org/fair-principles/fair-data-principles-explained/f1-meta-data-assigned-globally-unique-persistent-identifiers/}{\textbf{F1}. (Meta)data are assigned a globally unique and persistent identifier}

\href{https://www.go-fair.org/fair-principles/fair-data-principles-explained/f2-data-described-rich-metadata/}{\textbf{F2}. Data are described with rich metadata (defined by R1 below)}

\href{https://www.go-fair.org/fair-principles/f3-metadata-clearly-explicitly-include-identifier-data-describe/}{\textbf{F3}. Metadata clearly and explicitly include the identifier of the data they describe}

\href{https://www.go-fair.org/fair-principles/f4-metadata-registered-indexed-searchable-resource/}{\textbf{F4}. (Meta)data are registered or indexed in a searchable resource}

\textbf{Accessible}\\
Once the user finds the required data, she/he needs to know how can they be accessed, possibly including authentication and authorisation.

\href{https://www.go-fair.org/fair-principles/542-2/}{\textbf{A1}. (Meta)data are retrievable by their identifier using a standardised communications protocol}

\href{https://www.go-fair.org/fair-principles/a1-1-protocol-open-free-universally-implementable/}{\textbf{A1.1}~The protocol is open, free, and universally implementable}

\href{https://www.go-fair.org/fair-principles/a1-2-protocol-allows-authentication-authorisation-required/}{\textbf{A1.2}~The protocol allows for an authentication and authorisation procedure, where necessary}

\href{https://www.go-fair.org/fair-principles/a2-metadata-accessible-even-data-no-longer-available/}{\textbf{A2}. Metadata are accessible, even when the data are no longer available}

\textbf{Interoperable}\\
The data usually need to be integrated with other data. In addition, the data need to interoperate with applications or workflows for analysis, storage, and processing.

\href{https://www.go-fair.org/fair-principles/i1-metadata-use-formal-accessible-shared-broadly-applicable-language-knowledge-representation/}{\textbf{I1}. (Meta)data use a formal, accessible, shared, and broadly applicable language for knowledge representation.}

\href{https://www.go-fair.org/fair-principles/i2-metadata-use-vocabularies-follow-fair-principles/}{\textbf{I2}. (Meta)data use vocabularies that follow FAIR principles}

\href{https://www.go-fair.org/fair-principles/i3-metadata-include-qualified-references-metadata/}{\textbf{I3}. (Meta)data include qualified references to other (meta)data}

\textbf{Reusable}\\
The ultimate goal of FAIR is to optimise the reuse of data. To achieve this, metadata and data should be well-described so that they can be replicated and/or combined in different settings.

\href{https://www.go-fair.org/fair-principles/r1-metadata-richly-described-plurality-accurate-relevant-attributes/}{\textbf{R1}. (Meta)data are richly described with a plurality of accurate and relevant attributes}

\href{https://www.go-fair.org/fair-principles/r1-1-metadata-released-clear-accessible-data-usage-license/}{\textbf{R1.1}. (Meta)data are released with a clear and accessible data usage license}

\href{https://www.go-fair.org/fair-principles/r1-2-metadata-associated-detailed-provenance/}{\textbf{R1.2}. (Meta)data are associated with detailed provenance}

\href{https://www.go-fair.org/fair-principles/r1-3-metadata-meet-domain-relevant-community-standards/}{\textbf{R1.3}. (Meta)data meet domain-relevant community standards}

The principles refer to three types of entities: data (or any digital object), metadata (information about that digital object), and infrastructure. For instance, principle F4 defines that both metadata and data are registered or indexed in a searchable resource (the infrastructure component).
\end{quote}

\hypertarget{roles}{%
\section{Roles}\label{roles}}

\begin{quote}
or responsibilities. Who is required to do what? How are they held accountable, and by who? requirements for receiving resnet funding prequisites for sharing data on resnet platform
\end{quote}

\hypertarget{workflows}{%
\section{Workflows}\label{workflows}}

\hypertarget{internal-data}{%
\subsection{Internal Data}\label{internal-data}}

\hypertarget{external-data}{%
\subsection{External Data}\label{external-data}}

\begin{itemize}
\tightlist
\item
  Verify license
\item
  Retrieve and/or complete required metadata fields
\item
  Storage

  \begin{itemize}
  \tightlist
  \item
    Small to moderate datasets (\textless{} 2gb)

    \begin{itemize}
    \tightlist
    \item
      Upload to ResNet Data Portal
    \end{itemize}
  \item
    Large datasets (\textgreater{} 2gb)

    \begin{itemize}
    \tightlist
    \item
      Explore existing services
    \item
      Coordinate with ResNet data manager
    \end{itemize}
  \end{itemize}
\end{itemize}

\hypertarget{standards}{%
\chapter{Standards}\label{standards}}

\hypertarget{identifiers}{%
\section{Identifiers}\label{identifiers}}

\hypertarget{researchers-orcid}{%
\subsection{Researchers: ORCID}\label{researchers-orcid}}

\hypertarget{data-doi}{%
\subsection{Data: DOI}\label{data-doi}}

\hypertarget{physical-samples-igsn}{%
\subsection{Physical Samples: IGSN}\label{physical-samples-igsn}}

\hypertarget{repositories}{%
\section{Repositories}\label{repositories}}

\hypertarget{portage}{%
\subsection{Portage}\label{portage}}

\hypertarget{globus}{%
\subsection{GLOBUS}\label{globus}}

\hypertarget{resnet-data-portalgeonode}{%
\subsection{ResNet Data Portal/GeoNode}\label{resnet-data-portalgeonode}}

\hypertarget{metadata}{%
\section{Metadata}\label{metadata}}

\begin{quote}
retitle to documentation?
Define metadata and significance
\end{quote}

\hypertarget{iso-19115}{%
\subsection{ISO 19115}\label{iso-19115}}

\begin{quote}
Links to standard (\url{http://rd-alliance.github.io/metadata-directory/standards/iso-19115.html})
\end{quote}

\begin{quote}
Enumerate required fields
\end{quote}

\hypertarget{tools}{%
\subsection{Tools}\label{tools}}

\begin{quote}
Metadata creation and validation tools
\end{quote}

ESRI/ArcGIS:

Python:

\url{https://pycsw.org/}

\url{https://github.com/geopython/pygeometa}

R:

\url{https://github.com/eblondel/geometa}

Stand Alone:

Web:

\hypertarget{data-formats}{%
\section{Data Formats}\label{data-formats}}

\hypertarget{raster}{%
\subsection{Raster}\label{raster}}

\hypertarget{formats}{%
\subsubsection{Formats}\label{formats}}

\hypertarget{geotiff}{%
\paragraph{geotiff}\label{geotiff}}

\hypertarget{netcdf}{%
\paragraph{NetCDF}\label{netcdf}}

\hypertarget{vector}{%
\subsection{Vector}\label{vector}}

\hypertarget{shapefile}{%
\subsubsection{Shapefile}\label{shapefile}}

\hypertarget{geojson}{%
\subsubsection{GeoJSON}\label{geojson}}

\hypertarget{tabular}{%
\subsection{Tabular}\label{tabular}}

\hypertarget{csv}{%
\subsubsection{CSV}\label{csv}}

\hypertarget{resources}{%
\chapter{Resources}\label{resources}}

\url{https://journals.plos.org/ploscompbiol/article?id=10.1371/journal.pcbi.1005510}

\url{https://docs.computecanada.ca/wiki/Research_Data_Management}

\url{https://earthdata.nasa.gov/esdis/eso/standards-and-references/data-product-development-guide-for-data-producers}

\url{https://daac.ornl.gov/datamanagement/}

\url{https://www.usgs.gov/products/data-and-tools/data-management/data-management-plans}

  \bibliography{book.bib,packages.bib}

\end{document}
